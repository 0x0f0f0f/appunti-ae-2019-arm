\begin{defn}
    \textbf{Modi operativi del processore}.
    I processori, fra cui quelli ARM, hanno diversi modi di operazione. Nei processori in generale
    sono presenti i modi \verb|user| e \verb|kernel| (\textit{privileged}). In \verb|user| mode
    il codice in esecuzione non ha accesso diretto all'hardware o alla memoria, in \verb|kernel| mode
    il codice in esecuzione ha accesso senza restrizioni alle periferiche e alla memoria.

    In ARM sono presenti i modi:
    \begin{itemize}
        \item \verb|user|: esecuzione senza accesso all'hardware e alle periferiche.
        Ho accesso ad uno spazio di memoria ma non alla memoria completa
        \item \verb|fast interrupt|
        \item \verb|interrupt|
        \item \verb|supervisor|: accesso senza restrizioni al sistema, corrisponde all'esecuzione
        di istruzioni \verb|svc|.
        \item \verb|abort|
        \item \verb|system|
        \item \verb|undefined|
    \end{itemize}

    Nel modo utente sono presenti i 16 registri comuni, quando si passa a modi operativi diversi
    alcuni registri vengono duplicati, ad esempio quando passo al modo \verb|fast interrupt| vengono duplicati
    i registri da \verb|r0| a \verb|r8| per non interagire con quanto sta accadendo in modalità utente.
    In modalità \verb|supervisor| vengono duplicati i registri \verb|r13| e \verb|r14|.

    Si tiene traccia del modo corrente in una \textbf{parola di stato} del processore.
    Si abbrevia con \textit{CPSR} (Current Program Status Register).

\end{defn}


\begin{defn}
    \textbf{Cosa avviene al boot di un processore ARM}
    Per primo passo imposto \verb|PC| a 0 ed entro nella modalità CPU \verb|supervisor|.
    In tale modalità inizio ad eseguire le istruzioni alla locazione il \textbf{bootloader} carica il
    sistema operativo da disco e cambia la modalità di esecuzione a \verb|user|.
\end{defn}

\section{Homework Assembly ARM}

\begin{exrc}
    \textbf{Programma che calcola il prefisso di un vettore}
    \includecode[armn]{./asm/homework/prefix.s}{Programma che calcola il prefisso di un vettore}
\end{exrc}

\begin{exrc}
    \textbf{Programma che calcola la divisione con resto di due interi}
    \includecode[armn]{./asm/homework/divide.s}{Programma che calcola la divisione con resto di due interi}
\end{exrc}

\begin{exrc}
    \textbf{Programma che calcola la moltiplicazione di due interi}
    \includecode[armn]{./asm/homework/multiply.s}{Programma che calcola la moltiplicazione di due interi}
\end{exrc}

\begin{exrc}
    \textbf{Funzione che calcola il prodotto scalare di due vettori}
    \includecode[armn]{./asm/homework/scalarproduct.s}{Funzione che calcola il prodotto scalare di due vettori}
\end{exrc}

\begin{exrc}
    \textbf{Funzione che conta le occorrenze di un carattere in una stringa}
    \includecode[armn]{./asm/homework/countoccurrences.s}{Funzione che conta le occorrenze di un carattere in una stringa}
\end{exrc}

\begin{exrc}
    \textbf{Funzione che calcola l'elevamento a potenza di un intero, con base ed esponente positivi}
    \includecode[armn]{./asm/homework/power.s}{Funzione che calcola l'elevamento a potenza di un intero con base ed esponente positivi}
\end{exrc}

