\begin{defn}
    \textbf{Modi operativi del processore}.
    I processori, fra cui quelli ARM, hanno diversi modi di operazione. Nei processori in generale
    sono presenti i modi \texttt{user} e \texttt{kernel} (\textit{privileged}). In \texttt{user} mode
    il codice in esecuzione non ha accesso diretto all'hardware o alla memoria, in \texttt{kernel} mode
    il codice in esecuzione ha accesso senza restrizioni alle periferiche e alla memoria.

    In ARM sono presenti i modi:
    \begin{itemize}
        \item \texttt{user}: esecuzione senza accesso all'hardware e alle periferiche.
        Ho accesso ad uno spazio di memoria ma non alla memoria completa
        \item \texttt{fast interrupt}
        \item \texttt{interrupt}
        \item \texttt{supervisor}: accesso senza restrizioni al sistema, corrisponde all'esecuzione
        di istruzioni \texttt{svc}.
        \item \texttt{abort}
        \item \texttt{system}
        \item \texttt{undefined}
    \end{itemize}

    Nel modo utente sono presenti i 16 registri comuni, quando si passa a modi operativi diversi
    alcuni registri vengono duplicati, ad esempio quando passo al modo \texttt{fast interrupt} vengono duplicati
    i registri da \texttt{r0} a \texttt{r8} per non interagire con quanto sta accadendo in modalità utente.
    In modalità \texttt{supervisor} vengono duplicati i registri \texttt{r13} e \texttt{r14}.

    Si tiene traccia del modo corrente in una \textbf{parola di stato} del processore nell'unità di controllo.
    Si abbrevia con \textit{CPSR} (Current Program Status Register).

\end{defn}


\begin{defn}
    \textbf{Cosa avviene al boot di un processore ARM}
    Per primo passo imposto \texttt{PC} a 0 ed entro nella modalità CPU \texttt{supervisor}.
    In tale modalità inizio ad eseguire le istruzioni alla locazione il \textbf{bootloader} carica il
    sistema operativo da disco e cambia la modalità di esecuzione a \texttt{user}.
\end{defn}

\section{Homework Assembly ARM}

\begin{exrc}
    \textbf{Programma che calcola il prefisso di un vettore}
    \includecode[armn]{./asm/homework/prefix.s}{Programma che calcola il prefisso di un vettore}
\end{exrc}

\begin{exrc}
    \textbf{Programma che calcola la divisione con resto di due interi}
    \includecode[armn]{./asm/homework/divide.s}{Programma che calcola la divisione con resto di due interi}
\end{exrc}

\begin{exrc}
    \textbf{Programma che calcola la moltiplicazione di due interi}
    \includecode[armn]{./asm/homework/multiply.s}{Programma che calcola la moltiplicazione di due interi}
\end{exrc}

\begin{exrc}
    \textbf{Funzione che calcola il prodotto scalare di due vettori}
    \includecode[armn]{./asm/homework/scalarproduct.s}{Funzione che calcola il prodotto scalare di due vettori}
\end{exrc}

\begin{exrc}
    \textbf{Funzione che conta le occorrenze di un carattere in una stringa}
    \includecode[armn]{./asm/homework/countoccurrences.s}{Funzione che conta le occorrenze di un carattere in una stringa}
\end{exrc}

\begin{exrc}
    \textbf{Funzione che calcola l'elevamento a potenza di un intero, con base ed esponente positivi}
    \includecode[armn]{./asm/homework/power.s}{Funzione che calcola l'elevamento a potenza di un intero con base ed esponente positivi}
\end{exrc}

\section{Instruction Set ARMv7}

\centerfig{1}{arminstructionset}{Instruction Set ARMv7}

\begin{defn}
    \textbf{Rappresentazione di un'istruzione di Data Processing in una parola da 32 bit} \\
    Le istruzioni assembler vengono rappresentate in una parola da 32 bit.

    La prima sezione di 4 bit della parola viene usata per i codici di condizionali (\textit{condition codes}).
    Vedi la tabella \ref{tab:codes}

    La seconda sezione di 2 bit contiene il tipo di istruzione:
    \begin{itemize}
        \item 00: istruzione operativa (\textit{Data Processing})
        \item 01: istruzione di memoria (\textit{load, store, etc.})
        \item 10: istruzione di salto (\textit{branch, branch and link})
        \item 11: Data transfer, non vedremo queste istruzioni.
    \end{itemize}

    La prossima sezione, denominata \textit{FUNCT} è di 6 bit e determina l'istruzione da eseguire e gli operandi nel caso ci troviamo in un'operazione di \textit{Data Processing}.
    Il primo bit di \textit{FUNCT}, detto \textit{I} determina il tipo dell'ultimo operando (operand 2).
    Se il bit \textit{I} è a 0 allora l'operando 2 è un registro, è un immediato altrimenti.
    I 4 bit centrali sono detti \textit{Operation Code} o \textbf{OPCODE} e determinano l'operazione da eseguire.
    Ad esempio, le istruzioni di tipo \textit{Data Processing} (Vedi figura \ref{fig:arminstructionset}) avranno i seguenti \textit{OPCODE}

    \def\opcode#1#2#3{\texttt{#1} & \small #2 & #3\\}
    \begin{table}[H]
	\centering
	\caption{Tabella degli OPCODE delle principali operazioni di data processing}
	\label{tab:opcodes}
	\begin{tabular}{l@{\hspace{5mm}}ll}
	%
	\multicolumn{3}{c}{\bfseries \emph{opcode}: Data Processing Operation Codes } \\
	\opcode{Assembler Mnemonic}{OPCODE}{Action}
	\hline
	%
    \opcode{AND}{0000}{op1 AND op2}
    \opcode{EOR}{0001}{op1 EOR op2}
    \opcode{SUB}{0010}{op1 - op2}
    \opcode{RSB}{0011}{op2 - op1}
    \opcode{ADD}{0100}{op1 + op2}
    \opcode{ADC}{0101}{op1 + op2 + carry}
    \opcode{SBC}{0110}{op1 - op2 + carry - 1}
    \opcode{RSC}{0111}{op2 - op1 + carry - 1}
    \opcode{TST}{1000}{like AND but result is not written}
    \opcode{TEQ}{1001}{like EOR but result is not written}
    \opcode{CMP}{1010}{like SUB but result is not written}
    \opcode{CMN}{1011}{like ADD but result is not written}
    \opcode{ORR}{1100}{op1 OR op2}
    \opcode{MOV}{1101}{move op2 in the register op1}
    \opcode{BIC}{1110}{op1 AND NOT op2 (bit clear)}
    \opcode{MVN}{1111}{move NOT op2 in the register op1}
    \hline
    \end{tabular}
    \end{table}
    L'ultimo bit rimanente nella sezione \textit{FUNCT} è detto \textit{SET}. Se impostato ad 1 allora verranno impostati i bit di flag alla fine dell'esecuzione dell'istruzione.
    Ne seguono due sezioni da 4 bit che indicano il registro di destinazione \texttt{rd} e
    il registro sorgente 1 \texttt{rx}.
    I rimanenti 12 bit vengono utilizzati per il secondo operando.
    Se il bit \textit{I} è ad 1, il secondo operando conterrà 4 bit di rotazione e 8 di valore immediato.
    Altrimenti, se \textit{I} è a 0, i 12 bit del secondo operando vengono interpretati nel seguente modo.
    Abbiamo due casi, dove il registro è shiftato di una costante, ed un secondo caso dove il registro è shiftato di
    un altro registro. Nel primo caso abbiamo 5 bit per il valore dello shift, 2 bit che indicano il tipo di shift (\textit{LSL,LSR,ASR,ROR}), 1 bit che indica se
    il valore dello shift è contenuto in una costante (0) o in un registro (1). I rimanenti 4 bit indicano il registro su cui applicare lo shift.
    Nel secondo caso (il valore di shift è contenuto in un registro) abbiamo 4 bit per indicare il registro da cui leggere il valore di shift, un bit sempre a 0,
    il tipo di shift come nel primo caso e 1 bit per il tipo di shift (in questo caso impostato ad 1), i rimanenti 4 bit, come prima, indicano il registro su cui applicare lo shift.
    % //TODO riprendi lezione 14 nov seconda parte.
\end{defn}

\begin{defn}
    \textbf{Rappresentazione di un istruzione di memoria}

    Nelle istruzioni di memoria (\textit{load e store}) i primi 4 bit sono sempre di codice condizionale (vedi Tabella \ref{tab:codes}).
    I seguenti 2 bit che determinano il tipo di operazione saranno 01.
    Il primo bit dei 6 bit \textit{FUNCT}, ovvero il bit \textit{I} che indica se l'ultimo operando è un valore immediato è negato
    a differenza delle istruzioni operative.
    Seguono 5 bit di controllo, rispettivamente

    \def\opcode#1#2#3{\texttt{#1} & \small #2 & \texttt{#3}\\}
    \begin{table}[htbp]
        \centering
        \label{tab:pubwl}
        \caption{Bit di controllo di un'istruzione di memoria}
        \begin{tabular}{l@{\hspace{5mm}}ll}
            \multicolumn{3}{c}{\bfseries \emph{opcode}: Memory Data Transfer Instruction Control Codes} \\
            \opcode{bit}{meaning}{example}
            \hline
            \opcode{P}{if set, use pre-index}{ldr r0, [r1, \#4]}
            \opcode{U}{if set, use negative offset}{ldr r0, [r1, -r2]}
            \opcode{B}{if set, load a single byte instead of 4}{ldrb r0, [r1]}
            \opcode{W}{if set, write back the index+offset in the index register}{}
            \opcode{}{used combined with P, such that}{}
            \opcode{PW = 00}{Use post-index}{ldr r0, [r1], \#4}
            \opcode{PW = 01}{Not used}{}
            \opcode{PW = 10}{Use offset normally}{}
            \opcode{PW = 11}{Use pre-index}{ldr r0, [r1, \#4]!}
            \opcode{L}{if set, load, otherwise store}{ldr r0, [r1]}
            \opcode{}{}{str r0, [r1]}
        \end{tabular}
    \end{table}

    Seguono poi i due registri (4 bit), destinazione e base.
    L'ultimo operando comprende i 12 bit rimanenti e indica un valore immediato o un ulteriore registro.
\end{defn}

\begin{defn}
    \textbf{Struttura delle istruzioni di salto}
    Nelle istruzioni di salto i primi 4 bit sono sempre di condizione, i bit di operazione sono settati a 01.
    Seguono due bit che indicano il tipo di salto, vedremo solo le configurazioni 10 (\textit{branch}) e 11 (\textit{branch and link}).
    Nei rimanenti 24 bit è contenuto il valore immediato di offset (positivo o negativo) da sommare al valore corrente del registro \textit{Program Counter}.
\end{defn}

\begin{defn}
    \textbf{Istruzioni thumb}
    Nella parola di stato \textit{CPSR} è presente un bit che indica se il processore
    sta operando in modalità \textit{thumb}, realizzata da ARM per realizzare le stesse istruzioni utilizzando meno bit,
    e di conseguenza risparmiando memoria ed energia.
    Nella modalità thumb sono disponibili soltanto 8 registri generali, con in più \texttt{SP} e \texttt{PC}.
    In questo modo, servono solo 3 bit per indicare un registro.
    Per le istruzioni operative, se il registro di destinazione e il registro sorgente sono lo stesso,
    l'istruzione viene codificata indicando solo il registro destinazione, risparmiando 3 bit.
    La lunghezza degli operandi immediati viene ridotta se possibile.
    In modalità thumb, l'esecuzione condizionale è possibile solo per le istruzioni di salto

\end{defn}

\section{Istruzioni in virgola mobile}

\begin{defn}
    \textbf{Registri in virgola mobile}
    Mentre i registri \texttt{r0-r15} si usano per mantenere valori interi in complemento a 2 da 32 bit,
    esistono 32 registri da 32 bit per i valori in virgola mobile (\textit{float}), \texttt{s0-s31}.
    Possono essere usati due a due (\texttt{s0-s1, s2-s3}) come 16 registri a doppia precisione (\textit{double}), \texttt{d0-d15}.
\end{defn}

\begin{defn}
    \textbf{Istruzioni su registri virgola mobile}
    Per operare sui registri in virgola mobile, esistono i corrispondenti delle istruzioni
    operative regolari (sui valori interi), a cui va prefisso il carattere \texttt{v}.
    Ad esempio: \texttt{vabs, vadd, vmov, vsub, vcmp, ...}
    Si specifica anche la lunghezza del valore sul quale si desidera operare, ad
    esempio \texttt{vadd.f32 s0, s1}.


    Vale anche per le istruzioni di memoria: \texttt{VLDR, VSTR}
\end{defn}


% //TODO Tail recursive functions