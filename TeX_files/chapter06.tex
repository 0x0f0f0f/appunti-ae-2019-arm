\part{}
\chapter{Assembler ARM e microarchitettura}

\section{Introduzione all'Assembler ARM}

Abbiamo visto un ciclo che viene eseguito dal processore.
\begin{lstlisting}
while(true) {
	preleva un istruzione (ASM)
	interpreta
	esegui
}
\end{lstlisting}

\begin{defn}
Prelevare un istruzione significa estrarre dalla memoria una parola (configurazione di 0 e 1) contenente un istruzione \textbf{Assembly}. L'Assembly, spesso abbreviato \textbf{asm} è un linguaggio di programmazione a basso livello con una forte corrispondenza fra il linguaggio e le istruzioni in linguaggio macchina di una particolare architettura. Vedremo il linguaggio Assembler per ARM, architettura acronimo di Advanced Risc Machines, dove Risc sta per Reduced Instruction Set Computer. Risc utilizza molte meno istruzioni di CISC, più comune nelle architetture X86 e X86\_64. I principi di Risc sono:
\begin{enumerate}
	\item La regolarità $ \implies $ semplicità
	\item Supporto al caso più frequente
	\item "Piccolo è bello"
	\item Un buon risultato è frutto di un buon compromesso
\end{enumerate}

Il linguaggio asm viene compilato a linguaggio macchina. asm è un linguaggio mnemonico più vicino possibile alle istruzioni macchina. Una volta compilato e caricato in memoria, le istruzioni di un programma Assembler risiedono in dei registri.
\end{defn}

\begin{defn}
Nell'architettura \textbf{Von Neumann} è presente un \textbf{Program Counter} (abbreviato con pc), un contatore (registro). Nel caso dell'architettura ARMv7 (32 bit) il pc è un registro generale in cui è abilitata la scrittura. In ARMv8 (64 bit) non è più possibile utilizzare il pc come un registro generale. Il contenuto del pc indica l'indirizzo di memoria dove risiede l'indirizzo dell'istruzione corrente eseguita dal ciclo del processore, e dev'essere aggiornato per contenere l'indirizzo della prossima istruzione alla fine del ciclo.
\end{defn}

\begin{defn}
\textbf{Tipi di istruzioni Assembler:}
Vedremo istruzioni \textbf{aritmetico-logiche}, istruzioni di \textbf{load/store}, istruzioni di \textbf{controllo di flusso} ed alcune istruzioni \textbf{speciali}.
\end{defn}

\section{Istruzioni Assembler ARMv7}

\begin{note}
Il linguaggio Assembler ARM è radicalmente diverso dai linguaggi Assembler della famiglia X86! Vi sono diversi tipi di istruzion ARM asm, vediamo i tipi di operandi (parametri) che le istruzioni primitive ricevono.
\end{note}

\begin{defn}
\textbf{Sintassi delle istruzioni ARMv7:}

\begin{note}
	Le istruzioni sono case-insensitive.
\end{note}

\begin{enumerate}
	\item \textbf{Istruzioni a singolo operando (MOV, MOVN)}
	\begin{Verbatim}
	<opcode>{cond}{flags} Rd, <Op2>
	\end{Verbatim}

	\item \textbf{Istruzioni che non producono un risultato (CMP, CMN, TEQ, TST)}
	\begin{Verbatim}
	<opcode>{cond} Rn, <Op2>
	\end{Verbatim}

	\item \textbf{Altre istruzioni (AND,EOR,SUB,RSB,ADD,ADC,SBC,RSC,ORR,BIC)}
	\begin{Verbatim}
	<opcode>{cond}{flags} Rd, Rn, <Op2>
	\end{Verbatim}
\end{enumerate}


\begin{itemize}
	\item \texttt{<opcode>} è l'istruzione da eseguire
	\item \texttt{{cond}} Un codice condizionale di due lettere (opzionale).
	\item \texttt{{flags}} Flag opzionali aggiuntivi fra cui \texttt{S}, che è implicito per le istruzioni CMP, CMN, TEQ, TST.
	\item \texttt{OP2} Un secondo operando flessibile (opzionale), ha la forma \texttt{Rm{,<shift>}} oppure \texttt{<\#imm>}
	\item \texttt{Rd} Il registro di destinazione dell'operazione
	\item \texttt{Rn,Rm} Registri sorgente.
	\item \texttt{<\#imm>} Un valore immediato (se rappresentabile).
	\item \texttt{<shift>} Un operazione di shift della forma \texttt{<shiftname> <register>} oppure della forma \\
	 \texttt{<shiftname> <\#imm>}, oppure soltanto \texttt{RRX}
	\item \texttt{<shiftname>} Opcode di un istruzione di shift (ASL, LSL, LSR, ASR, ROR)
	\item \texttt{@ commento} Si possono inserire commenti a fine riga.
\end{itemize}
\end{defn}

\def\code#1#2#3#4{\texttt{#1} & \small #2 & #3 & #4\\}
\begin{table}[H]
	\centering
	\caption{Tabella dei codici condizionali ARMv7}
	\label{tab:codes}
	\begin{tabular}{l@{\hspace{5mm}}lll}
	%
	\multicolumn{4}{c}{\bfseries \emph{cond}: condition code} \\
	\code{code}{meaning}{flags tested}{4 bit cond field}
	\hline
	%
	\code{AL \textrm{or omitted}}{always}{(ignored)}{1110}
	\code{EQ}{equal}{Z = 1}{0000}
	\code{NE}{not equal}{Z = 0}{0001}
	\code{CS}{carry set (same as \texttt{HS})}{C = 1}{0010}
	\code{CC}{carry clear (same as \texttt{LO})}{C = 0}{0011}
	\code{MI}{minus}{N = 1}{0100}
	\code{PL}{positive or zero}{N = 0}{0101}
	\code{VS}{overflow}{V = 1}{0110}
	\code{VC}{no overflow}{V = 0}{0111}
	\code{HS}{unsigned higher or same}{C = 1}{0010}
	\code{LO}{unsigned lower}{C = 0}{0011}
	\code{HI}{unsigned higher}{C = 1 $\wedge$ Z = 0}{1000}
	\code{LS}{unsigned lower or same}{C = 0 $\vee$ Z = 1}{1001}
	\code{GE}{signed greater than or equal}{N = V}{1010}
	\code{LT}{signed less than}{N $\neq$ V}{1011}
	\code{GT}{signed greater than}{Z = 0 $\wedge$ N = V}{1100}
	\code{LE}{signed less than or equal}{Z = 1 $\vee$ N $\neq$ V}{1101}
	\code{NV}{never}{}{1111}
	\hline
	\end{tabular}
\end{table}

\begin{defn}
\textbf{Registri ARMv7:}
Nelle cpu ARMv7 sono presenti soltanto 16 registri da 32 bit, della forma \texttt{rn} dove $ 0 \leq n < 16 $.

\begin{itemize}
	\item \texttt{r0}: Parametri temporanei o valori di ritorno. (Standard per i valori di ritorno di funzioni)
	\item \texttt{r1 - r3}: Argomenti, valori temporanei.
	\item \texttt{r4 - r12}: Valori temporanei.
	\item \texttt{r13}: Stack Pointer (SP).
	\item \texttt{r14}: Link register.
	\item \texttt{r15}: Program counter.
\end{itemize}
\end{defn}


\begin{defn}
	\textbf{Aliasing dei registri}

	Si possono usare degli alias per riferirsi ai registri utilizzando la direttiva
	\texttt{.req}. Ad esempio, inserendo
\begin{lstlisting}[style=arm]
	MYName .req r5
\end{lstlisting}
	Ci permette di utilizzare l'alias \texttt{MYNAME} al posto di \texttt{r5} nel codice che segue.
\end{defn}

\begin{defn}
\textbf{Operandi Immediati:}
Operandi che non risiedono in dei registri ma sono costanti
presenti all'interno della configurazione della parola, ad 
esempio \texttt{\#1} è costante per il numero 1. 
\texttt{\#0x1234} corrisponde in esadecimale al numero $ 4661 $
\end{defn}

\begin{defn}
\textbf{Operandi e istruzioni di Memoria:}
Gli operandi di memoria sono della forma:
\begin{lstlisting}[style=arm]
@ access memory by offset without altering rx
[rx, #imm]
[rx, rn]
[rx, rn, <shift>]
@ update rx by value then access memory
[rx, #imm]!
[rx, rn]!
[rx, rn, <shift>]!
@ access memory at rx then update rx by offset
[rx], #imm
[rx], rn
[rx], rn, <shift>
\end{lstlisting}

Tale operando va a leggere il contenuto della memoria all'indirizzo puntato dal registro \texttt{Rx}, a cui va aggiunto l'\texttt{offset}.

L'istruzione per caricare una parola dalla memoria in un registro è
\begin{lstlisting}[style=arm]
@ with pre-index r2 is unaltered
ldr r0,[r2, #0]
@ with post-increment r2 is incremented by 4 after load
ldr r0,[r2], #4
@ from a label
.data
x: .word 123
ldr r0, =x
@ treat x's content as an address and load it
ldr r0, [r0]
@ load a single byte with #1 byte post increment
ldrb r2, [r0], #1
\end{lstlisting}

Tale istruzione caricherà nel registro \texttt{r0} il contenuto dell'indirizzo puntato dal contenuto di \texttt{r2} a cui va aggiunto 0. L'offset è utile per accedere a strutture dati (memorizzate sequenzialmente) come \texttt{struct} e \texttt{array}, presenti nel linguaggio C.

Per caricare in un registro un operando immediato o il valore di un altro registro si usa
\begin{lstlisting}[style=arm]
mov rd, <src>
@ <src> is either a register or an immediate value
\end{lstlisting}

Per salvare il valore di un registro in memoria si usa
\begin{lstlisting}[style=arm]
str
\end{lstlisting}
\end{defn}

\begin{note}
La memoria di ARM, come microarchitettura, è indirizzata a byte (gli indirizzi corrispondono a singoli byte). Si può caricare un valore dalla memoria in un registro con
\begin{lstlisting}[style=arm]
mov r2, #0x400
ldr r0,[r2, #0]
\end{lstlisting}

All'interno del registro \texttt{r0} saranno caricati i 4 byte a partire dall'indirizzo 1024 (400 in esadecimale).
\end{note}

\begin{defn}
\textbf{Big Endian e Little Endian:} Se vogliamo caricare un numero da 4 byte (32 bit) in un registro dalla memoria, può sorgere il dubbio, in quale ordine vengono letti i byte? Nel metodo \textbf{Big Endian} carica partendo dal byte più significativo (quello più a sinistra), fino a quello meno significativo. Nel metodo \textbf{Little Endian}, il primo byte caricato è quello meno significativo (quello più a destra del registro/numero), mentre l'ultimo caricato è il byte più significativo. ARMv7 supporta entrambi i metodi di \textbf{endianness}.
\end{defn}

\begin{defn}
\textbf{Istruzioni Logiche Bitwise}

Realizza l'and bit per bit dei registri \texttt{r1} (maschera) e \texttt{r2} (sorgente), memorizzando in \texttt{rd}
\begin{lstlisting}[style=arm]
and rd, r1, r2
\end{lstlisting}

Realizza l'OR bit per bit dei registri \texttt{r1} e \texttt{r2}, memorizzando in \texttt{rd}
\begin{lstlisting}[style=arm]
orr rd, r1, r2
\end{lstlisting}

Realizza l'Exclusive OR bit per bit dei registri \texttt{r1} e \texttt{r2}, memorizzando in \texttt{rd}
\begin{lstlisting}[style=arm]
eor rd, r1, r2
\end{lstlisting}

Realizza l'Exclusive OR bit per bit dei registri \texttt{r1} e \texttt{r2}, memorizzando in \texttt{rd}
\begin{lstlisting}[style=arm]
eor rd, r1, r2
\end{lstlisting}

Imposta a 0 i bit di \texttt{r2} laddove nella posizione corrispondente della maschera \texttt{r1} vi sia un 1, memorizzando il risultato in \texttt{rd}
\begin{lstlisting}[style=arm]
bic rd, r1, r2
\end{lstlisting}
\end{defn}

\begin{defn}
\textbf{Bit di flag in ARMv7:}
In ARMv7 sono presenti 4 bit di flag di condizione (che possono essere alterati dopo l'esecuzione di un istruzione). Il flag \texttt{Z} indica se il risultato di un operazione è 0. Il flab \texttt{N} indica se il risultato dell'operazione è negativo. Il flag \texttt{C} indica se è presente un carry (riporto) sull'ultima cifra. Il flag \texttt{V} (overflow) indica se l'operazione ha dato overflow. I flag sono memorizzati in un registro detto "parola di stato", che non è accessibile come gli altri registri general purpose. Vi è presente anche un bit che indica l'endianness delle operazioni di memoria.
Solo alcune istruzioni impostano di default i set di condizione (come \texttt{CMP}). Per forzare un istruzione ad impostare i bit di flag di condizione si può passare il flag aggiuntivo \texttt{S}.
\end{defn}

\begin{defn}
\textbf{Istruzioni Aritmetiche}

Realizza la somma dei registri \texttt{r1} e \texttt{r2}, memorizzando il risultato in \texttt{rd}
\begin{lstlisting}[style=arm]
add rd, r1, r2
\end{lstlisting}

Realizza la somma dei registri \texttt{r1} e \texttt{r2} \textbf{con riporto}, memorizzando il risultato in \texttt{rd}
\begin{lstlisting}[style=arm]
adc rd, r1, r2
\end{lstlisting}
%	Todo spiega bit di carry

Realizza la sottrazione dei registri \texttt{r1 - r2}, memorizzando il risultato in \texttt{rd}
\begin{lstlisting}[style=arm]
sub rd, r1, r2
\end{lstlisting}

Realizza la sottrazione dei registri \texttt{r1 - r2} \textbf{con riporto}, memorizzando il risultato in \texttt{rd}
\begin{lstlisting}[style=arm]
sbc rd, r1, r2
\end{lstlisting}

Compare accetta solamente due operandi, realizza la sottrazione dei registri \texttt{r1 - r2} ed \textbf{imposta i bit di flag}. Viene utilizzato per realizzare esecuzione condizionale nei programmi, laddove le istruzioni successive utilizzino un codice condizionale.
\begin{lstlisting}[style=arm]
cmp r1, r2
\end{lstlisting}

Sottrazione "al contrario" (Reverse Subtraction), esegue l'operazione \texttt{r2 - r1} e memorizza il risultato in \texttt{rd}.
\begin{lstlisting}[style=arm]
sbc rd, r1, r2
\end{lstlisting}

\end{defn}

\begin{defn}
\textbf{Istruzioni di Shift (logico-aritmetiche)}

\begin{note}
	I registri passati per parametro alle istruzioni di shift non sono registri di destinazione ma contengono il numero parametro dello shift (di quante posizioni deve essere effettuato). Vedi il paragrafo sulla sintassi delle istruzioni Assembler.
\end{note}

Bitshift di \texttt{num} posizioni a sinistra. Semanticamente identico all'istruzione \texttt{ASL}. Equivalente alla moltiplicazione per la \texttt{num}-esima potenza di 2.
\begin{lstlisting}[style=arm]
@ come operando
lsl #imm
lsl rn
@ come istruzione
lsl rd, rn, #imm
\end{lstlisting}

Bitshift di \texttt{num} posizioni a destra. Equivalente alla divisione per la \texttt{num}-esima potenza di 2.
\begin{lstlisting}[style=arm]
@ come operando
lsr #imm
lsr rn
@ come istruzione
lsr rd, rn, #imm
\end{lstlisting}


Bitshift aritmetico di \texttt{num} posizioni a destra (preserva il bit di segno). Equivalente alla divisione per la \texttt{num}-esima potenza di 2.
\begin{lstlisting}[style=arm]
@ come operando
asr #imm
asr rn
@ come istruzione
asr rd, rn, #imm
\end{lstlisting}



Rotate Right: esegue uno shift bitwise a destra e re-inserisce il bit shiftato (meno significativo), nella posizione del bit più significativo (più a sinistra).
\begin{lstlisting}[style=arm]
@ come operando
ror #imm
ror rn
@ come istruzione
ror rd, rn, #imm
\end{lstlisting}

\end{defn}

% Todo cross-compilare e debuggare asm ARM su linux
% pseudo istruzioni da passare al compilatore per generare un eseguibile
% cross compila asm con gcc cross-compiler con -ggdb3 -static
% gcc -S per generare asm .S
% gcc-as per compilare .S
% avvia qemu con debugger bridge
% info registers su gdb

\begin{exmp}
\textbf{Program Counter:}


\begin{lstlisting}[style=arm]
mov r0, #15
add r0, r0, r0
\end{lstlisting}


Il program counter (\texttt{pc}) prima dell'esecuzione dell'istruzione \texttt{MOV} indicherà l'indirizzo di memoria dove è contenuta la parola dell'istruzione. Alla fine dell'esecuzione di \texttt{MOV} sarà incrementato di 4 posizioni (4 byte = 32 bit, dimensione di una parola).
\end{exmp}

\begin{defn}
\textbf{Istruzioni di salto}

L'istruzione branch "salta" all'istruzione contenuta in \texttt{pc}  a cui viene sommato un valore immediato.
Più semplicemente può saltare ad un istruzione contenuta in un etichetta.
Sono di fondamentale utilizzo i flag condizionali.
\begin{lstlisting}[style=arm]
B #imm		@ offset #imm is relative to pc
B label		@ branch to label
\end{lstlisting}

Salva il contenuto di \texttt{pc} in \texttt{LR} (registro 14) e "salta" all'istruzione contenuta in \texttt{pc + \#num}. Sono di fondamentale utilizzo i flag condizionali.
\begin{lstlisting}[style=arm]
BL #imm		@ offset #imm is relative to pc
BL label	@ branch to label
\end{lstlisting}
Si utilizza per realizzare l'equivalente di procedure e funzioni in asm.
\end{defn}

\section{Direttive, Pseudoistruzioni e Programmi}

\begin{defn}
\textbf{Direttive}
Le direttive non sono istruzioni asm, ma sono istruzioni per il compilatore (toolchain GNU) che compileranno a sezioni del programma. La sintassi delle direttive inizia con un simbolo "punto" \texttt{.}
\end{defn}

\begin{defn}
\textbf{Direttive di sezione del codice}

Inizia una nuova sezione di codice o dati.
\begin{lstlisting}[style=arm]
.section <section_name> {, "<flags>"}
\end{lstlisting}

Le sezioni \texttt{<section\_name>} nell'assembler ARM gnu possono essere:
\begin{itemize}
	\item \texttt{.text} una sezione di codice
	\item \texttt{.data} una sezione di dati inizializzati
	\item \texttt{.bss} una sezione di dati non inizializzati
	\item \texttt{.rodata} inizializza una sezione di dati (read only?)
\end{itemize}
\end{defn}

\begin{defn}
\textbf{Direttive di inizializzazione di dati}

Inserisce una lista di valori di parola da 32 bit nell'assembly.
\begin{lstlisting}[style=arm]
.word 123
.word 1,2,3,4
\end{lstlisting}

Inserisce una lista di byte nell'assembly.
\begin{lstlisting}[style=arm]
.byte <byte1> {, <byte2>, ...}
\end{lstlisting}

Inserisce una stringa di caratteri AScii nell'assembly.
\begin{lstlisting}[style=arm]
.ascii "<string>"
\end{lstlisting}

Come \texttt{.ascii}, inserisce una stringa di caratteri AScii nell'assembly, ma seguita da un byte 0.
\begin{lstlisting}[style=arm]
.asciz "<string>"
\end{lstlisting}

Riserva \texttt{number\_of\_bytes} byte in memoria, sono riempiti con byte 0.
\begin{lstlisting}[style=arm]
.space <number_of_bytes>
\end{lstlisting}
\end{defn}

\begin{defn}
\textbf{Chiamate al Sistema Operativo}
Come da C, possiamo fare chiamate al Sistema Operativo in Assembly. Si possono chiamare funzioni di librerie, che comprendono funzioni per i processi e funzioni per l'IO.

Realizza una chiamata di sistema, il cui codice è contenuto in \texttt{r7}. Le due istruzioni sono equivalenti.
\begin{lstlisting}[style=arm]
svc 0	@ "supervisor call"
swi 0	@ "software interrupt"
\end{lstlisting}
\end{defn}

\begin{exmp}
\texttt{write(fd, buffer, num\_byte);} è una funzione C che
stampa \texttt{num\_byte} del \texttt{buffer} all'interno di
\texttt{fd}, ovvero un \textit{file descriptor}.
La funzione della \textit{stdlib} C \texttt{printf}
che utilizza internamente \texttt{write}.
I file descriptor standard in sistemi Unix sono
\begin{itemize}
	\item 0: standard input o \texttt{stdin}
	\item 1: standard output o \texttt{stdout}
	\item 2: standard error o \texttt{stderr}
\end{itemize}

Il numero della syscall \texttt{write} è 4. Il numero della syscall exit è 1.
\end{exmp}

\begin{defn}
\textbf{Label}
I label identificano sezioni di codice.

\begin{lstlisting}[style=arm]
testlabel:
<instructions>
...
\end{lstlisting}

Per accedere all'indirizzo di memoria di un label \texttt{esempio} contenuto nella sezione \texttt{.data} si usa l'operando \texttt{=esempio}. Per i label nella sezione \texttt{.text} non va utilizzato il simbolo prefisso \texttt{=}.
\end{defn}

\begin{exmp}
\textbf{Hello World in Assembler ARM}
\includecode[armn]{./asm/helloworld/helloworld.s}{Hello World in Assembler ARM}
\end{exmp}

\begin{defn}
\textbf{If-then-else in asm ARM}
La direttiva \texttt{.if} rende il blocco di codice seguente condizionale. Analogamente esiste la direttiva \texttt{.else}. Una direttiva condizionale \texttt{.if} si termina con la direttiva \texttt{.endif}.
\begin{lstlisting}[style=armn]
.if <logical_expression>
	<do_something>
.else
	<do_something_else>
.endif
\end{lstlisting}

Vediamo come viene compilata ad Assembly una direttiva \texttt{if}, utilizzeremo quasi sempre le direttive e non condizioni testate manualmente.
\begin{lstlisting}[style=armn]
@ asm equivalent to C: if(x == y) {x++; y=2x} else x--;
	cmp r1, r2 	@ x and y are in r1 and r2, test condition
	bne else
then:
	add r1, r1, #1 	@ x++
	add r2, r1, r1	@ y=2x
	b cont
else:
	sub r1, r1, #1
cont:
	\dots			@ continue program
\end{lstlisting}
\end{defn}

\begin{defn}
\textbf{Ciclo for in asm ARM}

\begin{lstlisting}[style=armn]
@ asm equivalent to C: for(int i=0; i<4; i++)
	mov r0, #0			@ initialize counter
for:
	cmp #4, r0		@ test condition, sets flags
	beq endfor
	add r0, r0, #1 	@ step increment
endfor:
	... 			@ for is over, continue program
\end{lstlisting}
\end{defn}

\begin{defn}
\textbf{Switch-case in asm ARM}

\begin{lstlisting}
switch(x) {
	case 1: {x = 100; break;}
	case 2: {x = 200; break;}
	default: x = 300;
}
\end{lstlisting}

Realizziamo l'equivalente di questo snippet C in asm ARM

\begin{lstlisting}[style=armn]
case1:
	cmp r0, #1
	bne case2
	mov r0, #100
	b cont
case2:
	cmp r0, #2
	bne default
	mov r0, #200
	b cont
default:
	mov r0, #300
cont:
	... 	@ continue program
\end{lstlisting}
\end{defn}

\begin{exmp}
	\textbf{Allocare un vettore di interi}

\begin{lstlisting}[style=armn]
.data
a:	.word 1,2,3,4	@ array
n =	(.-a)/4
@ the number of bytes in the arr
@ divided by the number of bytes (at compile time)
@ in a word (4 bit integers)
@ gives us the len of array
\end{lstlisting}
\end{exmp}

\begin{exmp}
\textbf{Accedere ad un vettore}
\begin{lstlisting}
for(int i = 0; i < 5; i++) {
	y = x[i];
}
\end{lstlisting}

Realizziamo l'equivalente di questo snippet C in asm ARM

\begin{lstlisting}[style=armn]
mov r3, #0 			@ i = 0
loop:
	ldr r2, [r1, r3]	@ array's address is store in r1, use r3 as offset
				@ ... do something
	add r3, r3, #4 		@ i++ (memory offset is in bytes)
	cmp r3, #5 * #4		@ test i < 5 (mult by 4 bytes)
	bne loop
\end{lstlisting}

Oppure, semplificando utilizzando un post-indice per \texttt{ldr}:

\begin{lstlisting}[style=armn]
.data
array: .word 4,3,2,1    @ the array
length : (.-array) / 4	@ length

.text
.global main
	ldr r5, =length
	ldr r6, =array
loop:
	ldr r2, [r6], #4	@ load and add offset to r1 at the same time
	subs r5, r5, #1
	bne loop
\end{lstlisting}
\end{exmp}

\begin{exmp}
\textbf{Accedere ad un vettore, 2}
\begin{lstlisting}
for(int i = 0; i < 5; i++)
	x[i]++;
\end{lstlisting}

Realizziamo l'equivalente di questo snippet C in asm ARM


\begin{lstlisting}[style=armn]
.data
array: .word 4,3,2,1    @ the array
length : (.-array) / 4	@ length

.text
.global main
	ldr r5, =length
	ldr r6, =array
loop:
	ldr r2, [r6]
	add r2, r2, #1		@ x[i]++;
	str r2, [r6]		@ store r2
	add r6, r6, #4 		@ increment offset
	subs r5, r5, #1
	bne loop
\end{lstlisting}
\end{exmp}

\begin{exmp}
\textbf{Realizzare funzioni e chiamate di funzione}
Per realizzare una funzione in asm ARM si definisce una sezione di codice
con un label. Per "chiamare" la funzione si utilizza l'istruzione \texttt{bl nomefunzione}
(\textit{branch and link}) che eseguirà un \textit{branch} all'istruzione
contenuta nel label e salverà l'indirizzo dell'istruzione corrente all'interno del registro
\texttt{lr} o \texttt{r14} (\textit{link register}). Per ritornare dall'esecuzione
della funzione al codice chiamante si usa l'istruzione \texttt{bx lr} per ritornare
all'istruzione salvata nel \textit{link register}. Si può anche effettuare il return con una
condizione utilizzando ad esempio \texttt{moveq pc,lr} per impostare il registro
\textit{program counter} a \texttt{lr} se l'istruzione precedente ha impostato il flag
\textit{Zero} a 1.

Si possono usare i registri da \texttt{r0} a \texttt{r4} (escluso) per i parametri di una funzione.
\texttt{r0} è anche il valore di return. Il valore di ritorno della funzione va memorizzato in \texttt{r0}
prima del ritorno da essa.
Dal quarto parametro in poi, i parametri vanno salvati sullo stack con \texttt{push}.
Le funzioni della standard library C consumano direttamente i parametri necessari sullo stack.
\end{exmp}

\begin{defn}
	\textbf{Utilizzare lo Stack}
	Lo stack è una sezione di memoria del programma/processo, utilizzata con costrutti ed istruzioni
	particolari per essere equivalente alla struttura dati \textit{stack} comune in programmazione. Viene
	allocato quando viene creato il processo. Utilizziamo lo stack per memorizzare
	valori temporanei come le variabili locali di una funzione, o valori per i quali non vi è abbastanza
	spazio nei registri utilizzabili per i valori temporanei (ad esempio, gli argomenti dal quinto in poi in una chiamata a \texttt{printf}).
	Per interagire con lo stack utilizziamo le istruzioni \texttt{push} e \texttt{pop}, alias ad altre istruzioni di memoria per
	semplificarne l'accesso. Lo stack \textit{cresce} quando inseriamo una parola (32 bit) all'interno di esso, alla locazione
	contenuta nel registro \texttt{r13} o \texttt{sp} (\textit{stack pointer}).
\end{defn}


\begin{exmp}
\textbf{Esempio di utilizzo dello stack}
\begin{lstlisting}[style=armn]
main:
	mov 	r0, #2  @ set up r0
	push	{r0}    @ save r0 onto the stack
	mov 	r0, #3  @ overwrite r0
	pop 	{r0}    @ restore r0 to it's initial state
	mov	    r7, #1  @ finish the program
	svc		0
\end{lstlisting}
\end{exmp}

\begin{exmp}
\textbf{Chiamare funzioni C}
Per chiamare funzioni della standard library C o di una libreria C si può utilizzare la direttiva
\texttt{.extern} per includere la funzione nel programma asm. Si possono chiamare
le funzioni rese visibili con \texttt{bl}, come una normale funzione asm.
\end{exmp}

\begin{exmp}
\textbf{Hello world con funzioni della stdlib C}
\begin{lstlisting}[style=armn]
@ Make the glibc symbols visible.
.extern exit, puts
.data
	msg: .asciz "hello world"
.text
.global main
main:
	@r0 is the first argument.
	ldr r0, =msg
	bl puts
	mov r0, #0
	bl exit
\end{lstlisting}
\end{exmp}

\begin{exmp}
\textbf{Programma per stampare un array con printf}
\includecode[armn]{asm/printarray/printarray.s}{printarray.s}
\end{exmp}

\begin{exmp}
	\textbf{Equivalente della funzione main e argomenti a linea di comando}
	All'avvio della sezione main del codice ARM asm GNU, in
	\texttt{r0} sarà contenuto il numero di argomenti equivalente ad \texttt{argc}
	e in \texttt{r1} sarà contenuto l'indirizzo di memoria dove l'equivalente di \texttt{argv} è contenuto.
\end{exmp}
`'
\begin{defn}
    \textbf{Modi operativi del processore}.
    I processori, fra cui quelli ARM, hanno diversi modi di operazione. Nei processori in generale
    sono presenti i modi \verb|user| e \verb|kernel| (\textit{privileged}). In \verb|user| mode
    il codice in esecuzione non ha accesso diretto all'hardware o alla memoria, in \verb|kernel| mode
    il codice in esecuzione ha accesso senza restrizioni alle periferiche e alla memoria.

    In ARM sono presenti i modi:
    \begin{itemize}
        \item \verb|user|: esecuzione senza accesso all'hardware e alle periferiche.
        Ho accesso ad uno spazio di memoria ma non alla memoria completa
        \item \verb|fast interrupt|
        \item \verb|interrupt|
        \item \verb|supervisor|: accesso senza restrizioni al sistema, corrisponde all'esecuzione
        di istruzioni \verb|svc|.
        \item \verb|abort|
        \item \verb|system|
        \item \verb|undefined|
    \end{itemize}

    Nel modo utente sono presenti i 16 registri comuni, quando si passa a modi operativi diversi
    alcuni registri vengono duplicati, ad esempio quando passo al modo \verb|fast interrupt| vengono duplicati
    i registri da \verb|r0| a \verb|r8| per non interagire con quanto sta accadendo in modalità utente.
    In modalità \verb|supervisor| vengono duplicati i registri \verb|r13| e \verb|r14|.

    Si tiene traccia del modo corrente in una \textbf{parola di stato} del processore.
    Si abbrevia con \textit{CPSR} (Current Program Status Register).

\end{defn}


\begin{defn}
    \textbf{Cosa avviene al boot di un processore ARM}
    Per primo passo imposto \verb|PC| a 0 ed entro nella modalità CPU \verb|supervisor|.
    In tale modalità inizio ad eseguire le istruzioni alla locazione il \textbf{bootloader} carica il
    sistema operativo da disco e cambia la modalità di esecuzione a \verb|user|.
\end{defn}
