\chapter{Input/Output}

\begin{defn}
    \textbf{I/O} \\
    Un dispositivo I/O è un qualsiasi oggetto collegato al calcolatore che
    interagisce con esso e con il "\textit{mondo esterno}". Esistono diversi
    tipi di gestione dell'I/O (come viene gestito dal processore: \textbf{Memory
    Mapped I/O, DMA, Interrupt}.
\end{defn}

\begin{defn}
    \textbf{Memory Mapped I/O} \\
    Il Memory Mapped I/O è un metodo di scambio di dati fra una CPU e le
    periferiche collegate ad essa.
    L'I/O Memory-mapped usa lo stesso bus di indirizzi per indirizzare la
    memoria e i devices I/O. E le istruzioni della CPU usate sono le stesse per
    accedere ai dispositivi ed alla memoria.
    Viene riservata una sezione degli indirizzi codificabili (ad esempio 1GB
    su 4GB disponibili in un'architettura a 32 bit) per distinguere una sezione
    dedicata alla memoria (generalmente la sezione più grande) ed una sezione
    dedicata ai dispositivi I/O. Si aggiunge anche un codificatore al datapath
    che seleziona, in base all'indirizzo, se un'operazione di load e store va
    indirizzata alla MMU o ad un dispositivio I/O.
    % //TODO schema
\end{defn}

% //TODO ciclo while di lettura e scrittura da disco

\begin{defn}
    \textbf{Direct Memory Access (DMA)} \\
    Nel metodo DMA, la memoria ha le 4 linee (data in, data out, indirizzo e
    operazione) sia collegate al processore, sia collegate ai dispositivi I/O.
    Lo schema risulta simile al Memory Mapped I/O

    % //TODO schema
\end{defn}
